

	\documentclass[10pt,parskip=half,
	toc=sectionentrywithdots,
	bibliography=totocnumbered,
	captions=tableheading,numbers=noendperiod]{scrartcl}

    \usepackage[T1]{fontenc} % Nicer default font (+ math font) than Computer Modern for most use cases
    \usepackage{mathpazo}
    \usepackage{graphicx}
    \usepackage[skip=3pt]{caption}
    \usepackage{adjustbox} % Used to constrain images to a maximum size
    \usepackage[table]{xcolor} % Allow colors to be defined
    \usepackage{enumerate} % Needed for markdown enumerations to work
    \usepackage{amsmath} % Equations
    \usepackage{amssymb} % Equations
    \usepackage{textcomp} % defines textquotesingle
    % Hack from http://tex.stackexchange.com/a/47451/13684:
    \AtBeginDocument{%
        \def\PYZsq{\textquotesingle}% Upright quotes in Pygmentized code
    }
    \usepackage{upquote} % Upright quotes for verbatim code
    \usepackage{eurosym} % defines \euro
    \usepackage[mathletters]{ucs} % Extended unicode (utf-8) support
    \usepackage[utf8x]{inputenc} % Allow utf-8 characters in the tex document
    \usepackage{fancyvrb} % verbatim replacement that allows latex
    \usepackage{grffile} % extends the file name processing of package graphics
                         % to support a larger range
    % The hyperref package gives us a pdf with properly built
    % internal navigation ('pdf bookmarks' for the table of contents,
    % internal cross-reference links, web links for URLs, etc.)
    \usepackage{hyperref}
    \usepackage{longtable} % longtable support required by pandoc >1.10
    \usepackage{booktabs}  % table support for pandoc > 1.12.2
    \usepackage[inline]{enumitem} % IRkernel/repr support (it uses the enumerate* environment)
    \usepackage[normalem]{ulem} % ulem is needed to support strikethroughs (\sout)
                                % normalem makes italics be italics, not underlines

    \usepackage{translations}
	\usepackage{microtype} % improves the spacing between words and letters
	\usepackage{placeins} % placement of figures
    % could use \usepackage[section]{placeins} but placing in subsection in command section
	% Places the float at precisely the location in the LaTeX code (with H)
	\usepackage{float}
	\usepackage[colorinlistoftodos,obeyFinal,textwidth=.8in]{todonotes} % to mark to-dos
	% number figures, tables and equations by section
	\usepackage{chngcntr}
	% header/footer
	\usepackage[footsepline=0.25pt]{scrlayer-scrpage}

	% bibliography formatting
	\usepackage[numbers, square, super, sort&compress]{natbib}
	% hyperlink doi's
	\usepackage{doi}

    % define a code float
    \usepackage{newfloat} % to define a new float types
    \DeclareFloatingEnvironment[
        fileext=frm,placement={!ht},
        within=section,name=Code]{codecell}
    \DeclareFloatingEnvironment[
        fileext=frm,placement={!ht},
        within=section,name=Text]{textcell}
    \DeclareFloatingEnvironment[
        fileext=frm,placement={!ht},
        within=section,name=Text]{errorcell}

    \usepackage{listings} % a package for wrapping code in a box
    \usepackage[framemethod=tikz]{mdframed} % to fram code

% Pygments definitions

\makeatletter
\def\PY@reset{\let\PY@it=\relax \let\PY@bf=\relax%
    \let\PY@ul=\relax \let\PY@tc=\relax%
    \let\PY@bc=\relax \let\PY@ff=\relax}
\def\PY@tok#1{\csname PY@tok@#1\endcsname}
\def\PY@toks#1+{\ifx\relax#1\empty\else%
    \PY@tok{#1}\expandafter\PY@toks\fi}
\def\PY@do#1{\PY@bc{\PY@tc{\PY@ul{%
    \PY@it{\PY@bf{\PY@ff{#1}}}}}}}
\def\PY#1#2{\PY@reset\PY@toks#1+\relax+\PY@do{#2}}

\expandafter\def\csname PY@tok@w\endcsname{\def\PY@tc##1{\textcolor[rgb]{0.73,0.73,0.73}{##1}}}
\expandafter\def\csname PY@tok@c\endcsname{\let\PY@it=\textit\def\PY@tc##1{\textcolor[rgb]{0.25,0.50,0.50}{##1}}}
\expandafter\def\csname PY@tok@cp\endcsname{\def\PY@tc##1{\textcolor[rgb]{0.74,0.48,0.00}{##1}}}
\expandafter\def\csname PY@tok@k\endcsname{\let\PY@bf=\textbf\def\PY@tc##1{\textcolor[rgb]{0.00,0.50,0.00}{##1}}}
\expandafter\def\csname PY@tok@kp\endcsname{\def\PY@tc##1{\textcolor[rgb]{0.00,0.50,0.00}{##1}}}
\expandafter\def\csname PY@tok@kt\endcsname{\def\PY@tc##1{\textcolor[rgb]{0.69,0.00,0.25}{##1}}}
\expandafter\def\csname PY@tok@o\endcsname{\def\PY@tc##1{\textcolor[rgb]{0.40,0.40,0.40}{##1}}}
\expandafter\def\csname PY@tok@ow\endcsname{\let\PY@bf=\textbf\def\PY@tc##1{\textcolor[rgb]{0.67,0.13,1.00}{##1}}}
\expandafter\def\csname PY@tok@nb\endcsname{\def\PY@tc##1{\textcolor[rgb]{0.00,0.50,0.00}{##1}}}
\expandafter\def\csname PY@tok@nf\endcsname{\def\PY@tc##1{\textcolor[rgb]{0.00,0.00,1.00}{##1}}}
\expandafter\def\csname PY@tok@nc\endcsname{\let\PY@bf=\textbf\def\PY@tc##1{\textcolor[rgb]{0.00,0.00,1.00}{##1}}}
\expandafter\def\csname PY@tok@nn\endcsname{\let\PY@bf=\textbf\def\PY@tc##1{\textcolor[rgb]{0.00,0.00,1.00}{##1}}}
\expandafter\def\csname PY@tok@ne\endcsname{\let\PY@bf=\textbf\def\PY@tc##1{\textcolor[rgb]{0.82,0.25,0.23}{##1}}}
\expandafter\def\csname PY@tok@nv\endcsname{\def\PY@tc##1{\textcolor[rgb]{0.10,0.09,0.49}{##1}}}
\expandafter\def\csname PY@tok@no\endcsname{\def\PY@tc##1{\textcolor[rgb]{0.53,0.00,0.00}{##1}}}
\expandafter\def\csname PY@tok@nl\endcsname{\def\PY@tc##1{\textcolor[rgb]{0.63,0.63,0.00}{##1}}}
\expandafter\def\csname PY@tok@ni\endcsname{\let\PY@bf=\textbf\def\PY@tc##1{\textcolor[rgb]{0.60,0.60,0.60}{##1}}}
\expandafter\def\csname PY@tok@na\endcsname{\def\PY@tc##1{\textcolor[rgb]{0.49,0.56,0.16}{##1}}}
\expandafter\def\csname PY@tok@nt\endcsname{\let\PY@bf=\textbf\def\PY@tc##1{\textcolor[rgb]{0.00,0.50,0.00}{##1}}}
\expandafter\def\csname PY@tok@nd\endcsname{\def\PY@tc##1{\textcolor[rgb]{0.67,0.13,1.00}{##1}}}
\expandafter\def\csname PY@tok@s\endcsname{\def\PY@tc##1{\textcolor[rgb]{0.73,0.13,0.13}{##1}}}
\expandafter\def\csname PY@tok@sd\endcsname{\let\PY@it=\textit\def\PY@tc##1{\textcolor[rgb]{0.73,0.13,0.13}{##1}}}
\expandafter\def\csname PY@tok@si\endcsname{\let\PY@bf=\textbf\def\PY@tc##1{\textcolor[rgb]{0.73,0.40,0.53}{##1}}}
\expandafter\def\csname PY@tok@se\endcsname{\let\PY@bf=\textbf\def\PY@tc##1{\textcolor[rgb]{0.73,0.40,0.13}{##1}}}
\expandafter\def\csname PY@tok@sr\endcsname{\def\PY@tc##1{\textcolor[rgb]{0.73,0.40,0.53}{##1}}}
\expandafter\def\csname PY@tok@ss\endcsname{\def\PY@tc##1{\textcolor[rgb]{0.10,0.09,0.49}{##1}}}
\expandafter\def\csname PY@tok@sx\endcsname{\def\PY@tc##1{\textcolor[rgb]{0.00,0.50,0.00}{##1}}}
\expandafter\def\csname PY@tok@m\endcsname{\def\PY@tc##1{\textcolor[rgb]{0.40,0.40,0.40}{##1}}}
\expandafter\def\csname PY@tok@gh\endcsname{\let\PY@bf=\textbf\def\PY@tc##1{\textcolor[rgb]{0.00,0.00,0.50}{##1}}}
\expandafter\def\csname PY@tok@gu\endcsname{\let\PY@bf=\textbf\def\PY@tc##1{\textcolor[rgb]{0.50,0.00,0.50}{##1}}}
\expandafter\def\csname PY@tok@gd\endcsname{\def\PY@tc##1{\textcolor[rgb]{0.63,0.00,0.00}{##1}}}
\expandafter\def\csname PY@tok@gi\endcsname{\def\PY@tc##1{\textcolor[rgb]{0.00,0.63,0.00}{##1}}}
\expandafter\def\csname PY@tok@gr\endcsname{\def\PY@tc##1{\textcolor[rgb]{1.00,0.00,0.00}{##1}}}
\expandafter\def\csname PY@tok@ge\endcsname{\let\PY@it=\textit}
\expandafter\def\csname PY@tok@gs\endcsname{\let\PY@bf=\textbf}
\expandafter\def\csname PY@tok@gp\endcsname{\let\PY@bf=\textbf\def\PY@tc##1{\textcolor[rgb]{0.00,0.00,0.50}{##1}}}
\expandafter\def\csname PY@tok@go\endcsname{\def\PY@tc##1{\textcolor[rgb]{0.53,0.53,0.53}{##1}}}
\expandafter\def\csname PY@tok@gt\endcsname{\def\PY@tc##1{\textcolor[rgb]{0.00,0.27,0.87}{##1}}}
\expandafter\def\csname PY@tok@err\endcsname{\def\PY@bc##1{\setlength{\fboxsep}{0pt}\fcolorbox[rgb]{1.00,0.00,0.00}{1,1,1}{\strut ##1}}}
\expandafter\def\csname PY@tok@kc\endcsname{\let\PY@bf=\textbf\def\PY@tc##1{\textcolor[rgb]{0.00,0.50,0.00}{##1}}}
\expandafter\def\csname PY@tok@kd\endcsname{\let\PY@bf=\textbf\def\PY@tc##1{\textcolor[rgb]{0.00,0.50,0.00}{##1}}}
\expandafter\def\csname PY@tok@kn\endcsname{\let\PY@bf=\textbf\def\PY@tc##1{\textcolor[rgb]{0.00,0.50,0.00}{##1}}}
\expandafter\def\csname PY@tok@kr\endcsname{\let\PY@bf=\textbf\def\PY@tc##1{\textcolor[rgb]{0.00,0.50,0.00}{##1}}}
\expandafter\def\csname PY@tok@bp\endcsname{\def\PY@tc##1{\textcolor[rgb]{0.00,0.50,0.00}{##1}}}
\expandafter\def\csname PY@tok@fm\endcsname{\def\PY@tc##1{\textcolor[rgb]{0.00,0.00,1.00}{##1}}}
\expandafter\def\csname PY@tok@vc\endcsname{\def\PY@tc##1{\textcolor[rgb]{0.10,0.09,0.49}{##1}}}
\expandafter\def\csname PY@tok@vg\endcsname{\def\PY@tc##1{\textcolor[rgb]{0.10,0.09,0.49}{##1}}}
\expandafter\def\csname PY@tok@vi\endcsname{\def\PY@tc##1{\textcolor[rgb]{0.10,0.09,0.49}{##1}}}
\expandafter\def\csname PY@tok@vm\endcsname{\def\PY@tc##1{\textcolor[rgb]{0.10,0.09,0.49}{##1}}}
\expandafter\def\csname PY@tok@sa\endcsname{\def\PY@tc##1{\textcolor[rgb]{0.73,0.13,0.13}{##1}}}
\expandafter\def\csname PY@tok@sb\endcsname{\def\PY@tc##1{\textcolor[rgb]{0.73,0.13,0.13}{##1}}}
\expandafter\def\csname PY@tok@sc\endcsname{\def\PY@tc##1{\textcolor[rgb]{0.73,0.13,0.13}{##1}}}
\expandafter\def\csname PY@tok@dl\endcsname{\def\PY@tc##1{\textcolor[rgb]{0.73,0.13,0.13}{##1}}}
\expandafter\def\csname PY@tok@s2\endcsname{\def\PY@tc##1{\textcolor[rgb]{0.73,0.13,0.13}{##1}}}
\expandafter\def\csname PY@tok@sh\endcsname{\def\PY@tc##1{\textcolor[rgb]{0.73,0.13,0.13}{##1}}}
\expandafter\def\csname PY@tok@s1\endcsname{\def\PY@tc##1{\textcolor[rgb]{0.73,0.13,0.13}{##1}}}
\expandafter\def\csname PY@tok@mb\endcsname{\def\PY@tc##1{\textcolor[rgb]{0.40,0.40,0.40}{##1}}}
\expandafter\def\csname PY@tok@mf\endcsname{\def\PY@tc##1{\textcolor[rgb]{0.40,0.40,0.40}{##1}}}
\expandafter\def\csname PY@tok@mh\endcsname{\def\PY@tc##1{\textcolor[rgb]{0.40,0.40,0.40}{##1}}}
\expandafter\def\csname PY@tok@mi\endcsname{\def\PY@tc##1{\textcolor[rgb]{0.40,0.40,0.40}{##1}}}
\expandafter\def\csname PY@tok@il\endcsname{\def\PY@tc##1{\textcolor[rgb]{0.40,0.40,0.40}{##1}}}
\expandafter\def\csname PY@tok@mo\endcsname{\def\PY@tc##1{\textcolor[rgb]{0.40,0.40,0.40}{##1}}}
\expandafter\def\csname PY@tok@ch\endcsname{\let\PY@it=\textit\def\PY@tc##1{\textcolor[rgb]{0.25,0.50,0.50}{##1}}}
\expandafter\def\csname PY@tok@cm\endcsname{\let\PY@it=\textit\def\PY@tc##1{\textcolor[rgb]{0.25,0.50,0.50}{##1}}}
\expandafter\def\csname PY@tok@cpf\endcsname{\let\PY@it=\textit\def\PY@tc##1{\textcolor[rgb]{0.25,0.50,0.50}{##1}}}
\expandafter\def\csname PY@tok@c1\endcsname{\let\PY@it=\textit\def\PY@tc##1{\textcolor[rgb]{0.25,0.50,0.50}{##1}}}
\expandafter\def\csname PY@tok@cs\endcsname{\let\PY@it=\textit\def\PY@tc##1{\textcolor[rgb]{0.25,0.50,0.50}{##1}}}

\def\PYZbs{\char`\\}
\def\PYZus{\char`\_}
\def\PYZob{\char`\{}
\def\PYZcb{\char`\}}
\def\PYZca{\char`\^}
\def\PYZam{\char`\&}
\def\PYZlt{\char`\<}
\def\PYZgt{\char`\>}
\def\PYZsh{\char`\#}
\def\PYZpc{\char`\%}
\def\PYZdl{\char`\$}
\def\PYZhy{\char`\-}
\def\PYZsq{\char`\'}
\def\PYZdq{\char`\"}
\def\PYZti{\char`\~}
% for compatibility with earlier versions
\def\PYZat{@}
\def\PYZlb{[}
\def\PYZrb{]}
\makeatother

% ANSI colors
\definecolor{ansi-black}{HTML}{3E424D}
\definecolor{ansi-black-intense}{HTML}{282C36}
\definecolor{ansi-red}{HTML}{E75C58}
\definecolor{ansi-red-intense}{HTML}{B22B31}
\definecolor{ansi-green}{HTML}{00A250}
\definecolor{ansi-green-intense}{HTML}{007427}
\definecolor{ansi-yellow}{HTML}{DDB62B}
\definecolor{ansi-yellow-intense}{HTML}{B27D12}
\definecolor{ansi-blue}{HTML}{208FFB}
\definecolor{ansi-blue-intense}{HTML}{0065CA}
\definecolor{ansi-magenta}{HTML}{D160C4}
\definecolor{ansi-magenta-intense}{HTML}{A03196}
\definecolor{ansi-cyan}{HTML}{60C6C8}
\definecolor{ansi-cyan-intense}{HTML}{258F8F}
\definecolor{ansi-white}{HTML}{C5C1B4}
\definecolor{ansi-white-intense}{HTML}{A1A6B2}

% commands and environments needed by pandoc snippets
% extracted from the output of `pandoc -s`
\providecommand{\tightlist}{%
  \setlength{\itemsep}{0pt}\setlength{\parskip}{0pt}}
\DefineVerbatimEnvironment{Highlighting}{Verbatim}{commandchars=\\\{\}}
% Add ',fontsize=\small' for more characters per line
\newenvironment{Shaded}{}{}
\newcommand{\KeywordTok}[1]{\textcolor[rgb]{0.00,0.44,0.13}{\textbf{{#1}}}}
\newcommand{\DataTypeTok}[1]{\textcolor[rgb]{0.56,0.13,0.00}{{#1}}}
\newcommand{\DecValTok}[1]{\textcolor[rgb]{0.25,0.63,0.44}{{#1}}}
\newcommand{\BaseNTok}[1]{\textcolor[rgb]{0.25,0.63,0.44}{{#1}}}
\newcommand{\FloatTok}[1]{\textcolor[rgb]{0.25,0.63,0.44}{{#1}}}
\newcommand{\CharTok}[1]{\textcolor[rgb]{0.25,0.44,0.63}{{#1}}}
\newcommand{\StringTok}[1]{\textcolor[rgb]{0.25,0.44,0.63}{{#1}}}
\newcommand{\CommentTok}[1]{\textcolor[rgb]{0.38,0.63,0.69}{\textit{{#1}}}}
\newcommand{\OtherTok}[1]{\textcolor[rgb]{0.00,0.44,0.13}{{#1}}}
\newcommand{\AlertTok}[1]{\textcolor[rgb]{1.00,0.00,0.00}{\textbf{{#1}}}}
\newcommand{\FunctionTok}[1]{\textcolor[rgb]{0.02,0.16,0.49}{{#1}}}
\newcommand{\RegionMarkerTok}[1]{{#1}}
\newcommand{\ErrorTok}[1]{\textcolor[rgb]{1.00,0.00,0.00}{\textbf{{#1}}}}
\newcommand{\NormalTok}[1]{{#1}}

% Additional commands for more recent versions of Pandoc
\newcommand{\ConstantTok}[1]{\textcolor[rgb]{0.53,0.00,0.00}{{#1}}}
\newcommand{\SpecialCharTok}[1]{\textcolor[rgb]{0.25,0.44,0.63}{{#1}}}
\newcommand{\VerbatimStringTok}[1]{\textcolor[rgb]{0.25,0.44,0.63}{{#1}}}
\newcommand{\SpecialStringTok}[1]{\textcolor[rgb]{0.73,0.40,0.53}{{#1}}}
\newcommand{\ImportTok}[1]{{#1}}
\newcommand{\DocumentationTok}[1]{\textcolor[rgb]{0.73,0.13,0.13}{\textit{{#1}}}}
\newcommand{\AnnotationTok}[1]{\textcolor[rgb]{0.38,0.63,0.69}{\textbf{\textit{{#1}}}}}
\newcommand{\CommentVarTok}[1]{\textcolor[rgb]{0.38,0.63,0.69}{\textbf{\textit{{#1}}}}}
\newcommand{\VariableTok}[1]{\textcolor[rgb]{0.10,0.09,0.49}{{#1}}}
\newcommand{\ControlFlowTok}[1]{\textcolor[rgb]{0.00,0.44,0.13}{\textbf{{#1}}}}
\newcommand{\OperatorTok}[1]{\textcolor[rgb]{0.40,0.40,0.40}{{#1}}}
\newcommand{\BuiltInTok}[1]{{#1}}
\newcommand{\ExtensionTok}[1]{{#1}}
\newcommand{\PreprocessorTok}[1]{\textcolor[rgb]{0.74,0.48,0.00}{{#1}}}
\newcommand{\AttributeTok}[1]{\textcolor[rgb]{0.49,0.56,0.16}{{#1}}}
\newcommand{\InformationTok}[1]{\textcolor[rgb]{0.38,0.63,0.69}{\textbf{\textit{{#1}}}}}
\newcommand{\WarningTok}[1]{\textcolor[rgb]{0.38,0.63,0.69}{\textbf{\textit{{#1}}}}}

% Define a nice break command that doesn't care if a line doesn't already
% exist.
\def\br{\hspace*{\fill} \\* }

% Math Jax compatability definitions
\def\gt{>}
\def\lt{<}

    \setcounter{secnumdepth}{5}

    % Colors for the hyperref package
    \definecolor{urlcolor}{rgb}{0,.145,.698}
    \definecolor{linkcolor}{rgb}{.71,0.21,0.01}
    \definecolor{citecolor}{rgb}{.12,.54,.11}

\DeclareTranslationFallback{Author}{Author}
\DeclareTranslation{Portuges}{Author}{Autor}

\DeclareTranslationFallback{List of Codes}{List of Codes}
\DeclareTranslation{Catalan}{List of Codes}{Llista de Codis}
\DeclareTranslation{Danish}{List of Codes}{Liste over Koder}
\DeclareTranslation{German}{List of Codes}{Liste der Codes}
\DeclareTranslation{Spanish}{List of Codes}{Lista de C\'{o}digos}
\DeclareTranslation{French}{List of Codes}{Liste des Codes}
\DeclareTranslation{Italian}{List of Codes}{Elenco dei Codici}
\DeclareTranslation{Dutch}{List of Codes}{Lijst van Codes}
\DeclareTranslation{Portuges}{List of Codes}{Lista de C\'{o}digos}

\DeclareTranslationFallback{Supervisors}{Supervisors}
\DeclareTranslation{Catalan}{Supervisors}{Supervisors}
\DeclareTranslation{Danish}{Supervisors}{Vejledere}
\DeclareTranslation{German}{Supervisors}{Vorgesetzten}
\DeclareTranslation{Spanish}{Supervisors}{Supervisores}
\DeclareTranslation{French}{Supervisors}{Superviseurs}
\DeclareTranslation{Italian}{Supervisors}{Le autorit\`{a} di vigilanza}
\DeclareTranslation{Dutch}{Supervisors}{supervisors}
\DeclareTranslation{Portuguese}{Supervisors}{Supervisores}

\definecolor{codegreen}{rgb}{0,0.6,0}
\definecolor{codegray}{rgb}{0.5,0.5,0.5}
\definecolor{codepurple}{rgb}{0.58,0,0.82}
\definecolor{backcolour}{rgb}{0.95,0.95,0.95}

\lstdefinestyle{mystyle}{
    commentstyle=\color{codegreen},
    keywordstyle=\color{magenta},
    numberstyle=\tiny\color{codegray},
    stringstyle=\color{codepurple},
    basicstyle=\ttfamily,
    breakatwhitespace=false,
    keepspaces=true,
    numbers=left,
    numbersep=10pt,
    showspaces=false,
    showstringspaces=false,
    showtabs=false,
    tabsize=2,
    breaklines=true,
    literate={\-}{}{0\discretionary{-}{}{-}},
  postbreak=\mbox{\textcolor{red}{$\hookrightarrow$}\space},
}

\lstset{style=mystyle}

\surroundwithmdframed[
  hidealllines=true,
  backgroundcolor=backcolour,
  innerleftmargin=0pt,
  innerrightmargin=0pt,
  innertopmargin=0pt,
  innerbottommargin=0pt]{lstlisting}

 % Used to adjust the document margins
\usepackage{geometry}
\geometry{tmargin=1in,bmargin=1in,lmargin=1in,rmargin=1in,
nohead,includefoot,footskip=25pt}
% you can use showframe option to check the margins visually

	% ensure new section starts on new page
	\addtokomafont{section}{\clearpage}

    % Prevent overflowing lines due to hard-to-break entities
    \sloppy

    % Setup hyperref package
    \hypersetup{
      breaklinks=true,  % so long urls are correctly broken across lines
      colorlinks=true,
      urlcolor=urlcolor,
      linkcolor=linkcolor,
      citecolor=citecolor,
      }

    % ensure figures are placed within subsections
    \makeatletter
    \AtBeginDocument{%
      \expandafter\renewcommand\expandafter\subsection\expandafter
        {\expandafter\@fb@secFB\subsection}%
      \newcommand\@fb@secFB{\FloatBarrier
        \gdef\@fb@afterHHook{\@fb@topbarrier \gdef\@fb@afterHHook{}}}%
      \g@addto@macro\@afterheading{\@fb@afterHHook}%
      \gdef\@fb@afterHHook{}%
    }
    \makeatother

	% number figures, tables and equations by section
	\usepackage{chngcntr}
	\counterwithout{figure}{section}
	\counterwithout{table}{section}
	\counterwithout{equation}{section}
	\makeatletter
	\@addtoreset{table}{section}
	\@addtoreset{figure}{section}
	\@addtoreset{equation}{section}
	\makeatother
	\renewcommand\thetable{\thesection.\arabic{table}}
	\renewcommand\thefigure{\thesection.\arabic{figure}}
	\renewcommand\theequation{\thesection.\arabic{equation}}

    % align captions to left (indented)
	\captionsetup{justification=raggedright,
	singlelinecheck=false,format=hang,labelfont={it,bf}}

	% shift footer down so space between separation line
	\ModifyLayer[addvoffset=.6ex]{scrheadings.foot.odd}
	\ModifyLayer[addvoffset=.6ex]{scrheadings.foot.even}
	\ModifyLayer[addvoffset=.6ex]{scrheadings.foot.oneside}
	\ModifyLayer[addvoffset=.6ex]{plain.scrheadings.foot.odd}
	\ModifyLayer[addvoffset=.6ex]{plain.scrheadings.foot.even}
	\ModifyLayer[addvoffset=.6ex]{plain.scrheadings.foot.oneside}
	\pagestyle{scrheadings}
	\clearscrheadfoot{}
	\ifoot{\leftmark}
	\renewcommand{\sectionmark}[1]{\markleft{\thesection\ #1}}
	\ofoot{\pagemark}
	\cfoot{}

% clereref must be loaded after anything that changes the referencing system
\usepackage{cleveref}
\creflabelformat{equation}{#2#1#3}

% make the code float work with cleverref
\crefname{codecell}{code}{codes}
\Crefname{codecell}{code}{codes}
% make the text float work with cleverref
\crefname{textcell}{text}{texts}
\Crefname{textcell}{text}{texts}
% make the text float work with cleverref
\crefname{errorcell}{error}{errors}
\Crefname{errorcell}{error}{errors}

	\begin{document}

\hypertarget{jupyter-notebooks-with-stata}{%
\section{Jupyter Notebooks (with
STATA?!)}\label{jupyter-notebooks-with-stata}}

\hypertarget{what-are-jupyter-notebooks}{%
\subsection{What are Jupyter
Notebooks?}\label{what-are-jupyter-notebooks}}

\begin{itemize}
\tightlist
\item
  A way to do literate programming
\item
  Provide code and writing/analysis, on a language agnostic platform

  \begin{itemize}
  \tightlist
  \item
    Meaning that it is not restricted to just one language
  \item
    Currently there are so-called kernels for many languages
  \item
    Including Stata, Python, R, C, Golang, C++, Fortran and more coming!
  \end{itemize}
\item
  Uses the power of Markdown/Latex Math and Code to tell a story and
  provide an efficient workflow
\item
  Convert into several different formats including Latex, HTML,
  Presentations etc\ldots{}
\item
  The Jupyter engine is also available in other text editors such as
  Atom and VS Code.
\item
  And now available in STATA!
\end{itemize}

\hypertarget{under-the-hood}{%
\subsection{Under the Hood}\label{under-the-hood}}

\begin{itemize}
\tightlist
\item
  Jupyter Notebooks are written in python and are themselves a JSON
  document
\end{itemize}

\begin{itemize}
\tightlist
\item
  Which makes them suited for working on in a browser
\end{itemize}

\hypertarget{extensions}{%
\subsection{Extensions}\label{extensions}}

\begin{itemize}
\tightlist
\item
  Jupyter can be made to be a full featured IDE (Integrated Development
  Environment)
\item
  Which really means you can get all kinds of nifty things

  \begin{itemize}
  \tightlist
  \item
    Autocompletion
  \item
    Multi-cursor support
  \item
    Scratchpad
  \item
    Highlighting a selected word
  \item
    Translation
  \item
    Spellcheck
  \end{itemize}
\end{itemize}

\hypertarget{installing-extensions}{%
\subsection{Installing Extensions}\label{installing-extensions}}

\begin{itemize}
\tightlist
\item
  In order to do this, we need to go to our conda console and type:
\end{itemize}

\begin{verbatim}
 conda install -c conda-forge jupyter_contrib_nbextensions 
\end{verbatim}

\begin{itemize}
\tightlist
\item
  And restart Jupyter
\end{itemize}

\hypertarget{markdown}{%
\subsection{Markdown}\label{markdown}}

\begin{itemize}
\tightlist
\item
  Using the same idea as in markstat that Oscar showed you before.
\end{itemize}

\hypertarget{showing-math}{%
\subsection{Showing Math}\label{showing-math}}

\begin{itemize}
\tightlist
\item
  It is possible to show math

  \begin{itemize}
  \tightlist
  \item
    \(y_{it} = \alpha + \beta\cdot X\)
  \end{itemize}
\end{itemize}

\hypertarget{the-stata-kernel}{%
\subsection{The Stata Kernel}\label{the-stata-kernel}}

\begin{itemize}
\tightlist
\item
  This is a relatively new kernel that is implemented by Kyle Barron,
  Mauricio Cáceres, and other contributors

  \begin{itemize}
  \tightlist
  \item
    It provides the ability to run code and show graphics, which was
    previously unavailable for Stata in Jupyter.
  \end{itemize}
\item
  Ironically, even though we are using Stata in these presentations,
  there are other, free, open-source languages that are just as good (if
  not more powerful) for which dynamic documents have existed for over a
  decade.
\item
  As a small nudge towards getting you to try something like R or
  Python, here's an addendum that Kyle Barron wrote on this
  State\_kernel page:
\end{itemize}

\begin{quote}
As an ardent open-source advocate and someone who actively dislikes
using Stata, it somewhat pains me that my work creates value for a
proprietary, closed-source program. I hope that this program improves
research in a utilitarian way, and shows to new users the scope of the
open-source tools that have existed for upwards of a decade.
\end{quote}

\hypertarget{running-code}{%
\subsection{Running Code}\label{running-code}}

\begin{itemize}
\tightlist
\item
  In this case we will be using the Stata kernel, so we will have Stata
  running in the background.
\end{itemize}

\hypertarget{stata-kernel-magics}{%
\subsection{Stata Kernel Magics}\label{stata-kernel-magics}}

\begin{itemize}
\tightlist
\item
  Many Jupyter kernels have something called magics

  \begin{itemize}
  \tightlist
  \item
    A way to make certain actions easy without having to write too much
    code
  \item
    Stata has some magics that make things a little easier
  \end{itemize}
\end{itemize}

\hypertarget{browse-head-tail}{%
\subsection{\%browse, \%head, \%tail}\label{browse-head-tail}}

\begin{itemize}
\tightlist
\item
  This has the ability to choose varlist, the number of observations and
  with \texttt{if} statements as well
\end{itemize}

\hypertarget{html-and-latex}{%
\subsection{\%html and \%latex}\label{html-and-latex}}

\begin{itemize}
\tightlist
\item
  This allows the rendering of table during export into html or latex,
  as well as rendering in the notebook (with HTML only)
\end{itemize}

\hypertarget{help}{%
\subsection{\%help}\label{help}}

\begin{itemize}
\tightlist
\item
  You can use this to get a help file
\end{itemize}

\begin{figure}[H]\begin{center}\adjustimage{max size={0.9\linewidth}{0.9\paperheight},height=0.4\paperheight}{JupyterNotebooks_files/output_18_0.pdf}\end{center}\caption{A scatter plot}\label{fig:flabel}
    \end{figure}

\hypertarget{exporting}{%
\subsection{Exporting}\label{exporting}}

\hypertarget{using-ipypublish-to-get-publication-ready-pdfs}{%
\subsection{\texorpdfstring{Using \texttt{ipypublish} to Get Publication
Ready
PDFs}{Using ipypublish to Get Publication Ready PDFs}}\label{using-ipypublish-to-get-publication-ready-pdfs}}

\begin{itemize}
\tightlist
\item
  \texttt{ipypublish} is a utility developed for Jupyter Notebooks to
  make nice looking documents
\item
  To get this working, we need to use \texttt{pip}

  \begin{itemize}
  \tightlist
  \item
    In the conda console, type \texttt{pip\ install\ ipypublish}
  \item
    Hopefully it'll work
  \end{itemize}
\end{itemize}

\begin{itemize}
\tightlist
\item
  Doing this requires playing with the JSON code of a cell itself
  (called the metadata).
\item
  This allows a subsequent PDF output to be processed through latex,
  without any code cells and with figure and table environments.
\end{itemize}

\hypertarget{port-forwarding-and-setting-up-jupyter-to-work-on-a-server}{%
\subsection{Port-forwarding and setting up Jupyter to work on a
server}\label{port-forwarding-and-setting-up-jupyter-to-work-on-a-server}}

\begin{itemize}
\tightlist
\item
  Many people might have servers in their universities/organizations
  that are more powerful than a laptop.
\item
  Jupyter allows the ability to run a notebook locally (on your laptop
  screen), but using the power of the server.

  \begin{itemize}
  \tightlist
  \item
    This requires jupyter being installed on the server
  \item
    This isn't a difficult thing to do for a sysadmin, so it's worth
    finding out whether that's possible
  \end{itemize}
\end{itemize}

\hypertarget{setting-up-jupyter-on-a-server}{%
\subsection{Setting up jupyter on a
server}\label{setting-up-jupyter-on-a-server}}

\begin{itemize}
\tightlist
\item
  The first thing you need to do is log on to the server and start a
  jupyter instance:
\end{itemize}

\texttt{jupyter\ notebook\ -\/-no-browser\ -\/-port=8888}

\begin{itemize}
\item
  This tells the server to start an instance of jupyter, without a
  browser (we won't need it, nor can a server open up a browser window),
  in port 8888 (this will be important later)
\item
  For Mac users, you can use \texttt{ssh} to finish the process. Just
  type: \texttt{ssh\ username@host\ -L\ 8888:localhost:8888}
\item
  Which will forward your computer 8888 port, to the server's 8888 port.
\item
  For Windows, ssh also exists, but you will need to enable it.

  \begin{itemize}
  \tightlist
  \item
    head to Settings \textgreater{} Apps and click ``Manage optional
    features'' under Apps \& features.
  \item
    Click Add a Feature, and find OpenSSH
  \end{itemize}
\item
  Then use the same command as for Macs:
  \texttt{ssh\ username@host\ -L\ 8888:localhost:8888}
\item
  Then go to your browser:

  \begin{itemize}
  \tightlist
  \item
    \texttt{localhost:8888} and you should be taken to a Jupyter page
    and prompted for a token.
  \item
    You can find this token in the window where you started Jupyter on
    the server

    \begin{itemize}
    \tightlist
    \item
      Copy and paste this token into the prompt, and VOILA!
    \end{itemize}
  \end{itemize}
\item
  Now you have Jupyter running on your computer's browser window, but
  with the power of the server!
\end{itemize}

	\end{document}

